\documentclass[UTF8]{ctexart}
% \usepackage{amsfonts}
% \usepackage{amsmath}
% \usepackage{amssymb}
% \usepackage{amsthm}
% \usepackage{booktabs}
\usepackage{courier}
% \usepackage{float}
\usepackage{geometry}
\usepackage{graphicx}
\usepackage{hyperref}
% \usepackage{listings}
\geometry{left=2.54cm,right=2.54cm,top=2.18cm,bottom=3.18cm}

\begin{document}

\title{人脸表情识别——中期报告}
\author{无46\ \ 严靖凯\ \ 2014011192\\ 无46\ \ 王启睿\ \ 2014011179\\ 无46\ \ 黄秀峰\ \ 2014011193}
\maketitle

\section{概述}

对于人脸表情识别问题,目前学术界已有了较为充分的研究。对该问题的处理方法,大体可分为传统模式识别和神经网络两类。其中,传统模式识别方式的代表性文章有\cite{happy2015automatic,islam2016sention,wang2013feature,salmam2016facial}等,其中大多数采用SVM或决策树进行判决,神经网络方式的代表性文章有\cite{BarsoumICMI2016,khorrami2015deep}等。传统方法往往训练和运行速度很快,部分方法甚至可以应用于实时视频流作为输入等场景,对于质量高(清晰、光照合适、正脸)的图片也有很高的识别正确率;而神经网络方法对于一些传统方法难以完成的情形如很低的图片分辨率、不理想的光照、侧脸,或视频片段等。

由于本课程实验给了比较充裕的时间,因此我们考虑对这两种方法都进行一下尝试,一方面是为了验证究竟哪种特征提取与识别方法能获得更好的效果,另一方面也是为了对这些方法都进行练习,以熟悉其方法与技能。

目前,我们对神经网络和传统模式识别的方法都进行了考虑与尝试。以下将对它们进行详细介绍。

\section{方法一:CNN(VGG-13)网络特征提取}

我们本次实验提供的数据集上的数据量相对较小,而神经网络方法需要较大的训练集作为支撑,\cite{BarsoumICMI2016}提供的FER+数据集中有较多数据,因此我们选择%原谅她啊
在FER+数据集上复现文\cite{BarsoumICMI2016}的效果。

\begin{figure}[ht]
\includegraphics[width=\textwidth]{ferplus.png}
\caption{测试使用的CNN结构}\label{fig:ferplus}
\end{figure}

整个网络结构如上图所示,输入为为64x64灰度图片,其后的黄、绿、橙、蓝、灰色分别对应于卷积层、最大池化层、Dropout层、全连接+ ReLU层、Softmax层。该网络自VGG-13修改而来,以适应低分辨率的FER数据集。

训练得到的准确率在85\% 左右,与文中得到的结果基本一致。

事实上,该准确率很大程度上是受FER+数据集自身所限。FER+数据集使用$48\times48$的8bit灰度图片,且包含各种姿势(如侧脸、捂脸等),相比本次实验中最终需要测试的CK+、JAFFE等高分辨率正脸数据集而言,识别难度难度更大。针对前述测试数据集,下文提到的传统方法能够在更短时间、更小训练集下取得相仿甚至更高的正确率。

\section{方法二:特征提取+最近邻判定}

% 我们已经学习了libSVM的使用,并对libSVM自身的多类分类输出,针对我们的问题进行了改进,使得可以输出样本属于每一类的概率。
为了尝试更多不同的方法并观察效果,我们查阅了非深度学习的方法在人脸表情识别方面的表现。\cite{wang2013feature}提供了一种在JAFFE和Cohn-Kanade数据集上分别取得了93.97\%和95.86\%的准确率的方法,这种方法通过提取图片的WLD特征和HOG特征,并将直方图混合得到总特征,然后用卡方距离定义特征距离,再使用最近邻进行分类。这种方法给我们提供了一种思路,即通过提取特征并进行最近邻判定。目前我们尚未对这种方法做进一步实验。在接下来的实验中,我们将试着复现论文中提到的方法,使用WLD特征与HOG特征混合进行特征提取,并试着尝试其他的特征的效果(但预计单纯的更改特征很难得到比两种特征更高的正确率)。

\section{方法三:特征提取+SVM分类}

使用SVM也是一种经典的思路,但我们暂未对这方面展开更加深入的调研,因为我们最初并不看好传统的机器学习方法在这方面的表现,但前一种方法中提到的论文表现出的高正确率让我们觉得这种方法也是有可能得到非常好的效果的,所以决定在接下来的实验中进行尝试。在特征提取方面,将参考方法二,先尝试WLD+HOG看看效果,再尝试着使用其他在人脸方面有过较好表现的特征。

\section{进一步的工作}

CNN方面,我们考虑将网络换回VGG的$224\times224\times3$输入分辨率,或使用预训练好的VGG作为前级,以减小图片降采样和灰度化带来的信息损失,从而可能获得更高的正确率。

传统机器学习方法方面,我们将首先通过提取WLD+HOG特征,并用最近邻方法和SVM方法分别进行分类,再尝试其他在人脸检测方面比较常用的特征,对比不同特征和方法之间的效果。

最后,我们将综合以上提到的各种方法和思路,选择或融合得到一种最佳的判决器。

\section{实验心得与体会}

经过目前为止的实验,我们已经对人脸表情识别问题有了基本的认识,并调研与尝试了多种思路。目前的心得体会有:
\begin{itemize}
  \item 在进行文献调研初期,我们想当然地认为,神经网络的特征提取方法的准确率一定会明显优于传统的特征提取方法。然而在阅读多篇文章之后,我们发现事实并非这么绝对。在人脸表情识别这一具体的问题中,一些传统特征提取方法(如HOG)具有很强的针对性,比如可以分析出人的嘴角的方向,等等。相比之下,神经网络自动提取的特征则不一定具有这样好的性质。可见,一个方法并不一定在所有问题背景下都是最合适的。对于具体问题还需具体分析。
  \item 我们在媒体与认知课上学过神经网络的有关知识,在本次实验中将其付诸应用。我们通过阅读文献对,CNN有了更加直观的理解,同时学会了使用或调整现有的结构和成果,以实现特定的功能。这种能力对于计算机编程而言是至关重要的。
\end{itemize}

\bibliographystyle{unsrt}
\bibliography{midref}

\end{document}
