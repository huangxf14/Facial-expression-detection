\documentclass[UTF8]{ctexart}
% \usepackage{amsfonts}
% \usepackage{amsmath}
% \usepackage{amssymb}
% \usepackage{amsthm}
% \usepackage{booktabs}
\usepackage{courier}
% \usepackage{float}
\usepackage{geometry}
% \usepackage{graphicx}
% \usepackage{hyperref}
% \usepackage{listings}
\geometry{left=2.54cm,right=2.54cm,top=2.18cm,bottom=3.18cm}

\begin{document}

\title{人脸表情识别——中期报告}
\author{无46\ \ 严靖凯\ \ 2014011179\\ 无46\ \ 王启睿\ \ 2014011179\\ 无46\ \ 黄秀峰\ \ 2014011193}
\maketitle

\section{概述}

% 最好在里面加一些参考文献吧。
% 还要加点什么你们看吧~

对于人脸表情识别问题,目前学术界已有了较为充分的研究。对该问题的处理方法,大体可分为传统模式识别和深度学习两类。其中,传统模式识别方式的代表性文章有\cite{},深度学习方式的代表性文章有\cite{}。【比较一下两类方法的准确率,这个我不太懂,要不麻烦启睿写一下这里……?】
由于本课程实验给了比较充裕的时间,因此我们考虑对这两种方法都进行一下尝试,一方面是为了验证究竟哪种特征提取与识别方法能获得更好的效果,另一方面也是为了对这些方法都进行练习,以熟悉其方法与技能。

目前,我们着重考虑了两种方法,分别属于深度学习和传统模式时变的范畴,以下将对它们进行详细介绍。

\section{方法一:CNN-VGG7网络特征提取} % 这个标题取什么你们自己改一下

% 这里我想着写了一点,是介绍完理论之后的remark,可以参考我写的意思,当然也可以重新写。

% 我们发现助教所建议的几个数据集上的数据量相对较小,而在FER+数据集上有大量的数据。因此我们选择(原谅她啊 雾)在FER+数据集上先进行一个训练,观察初步的效果。”
% 我们发现,训练得到的准确率在85%左右,而这与 \cite{?} 中的95%的准确率有不小的差距。
% 事实上,该准确率很大程度上是受FER+数据集自身所限。其图像……


\section{方法二:HOG特征+SVM分类}

% 把从头到尾的理论写一下吧
% 然后就说具体的HOG特征提取代码还有待完善,还没有能整合起来测试

% 我们已经学习了libSVM的使用,并对libSVM自身的多类分类输出,针对我们的问题进行了改进,使得可以输出样本属于每一类的概率。

\section{进一步的工作}

% 这一块和下一块就麻烦您二位写一下了……

\section{实验心得与体会}


\bibliographystyle{unsrt}
\bibliography{midref}

\end{document}
